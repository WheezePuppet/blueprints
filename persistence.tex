
\chapter{Persistence and hydration}

Often, we want some of our objects to maintain their existence between
executions of the program. They're intended to be durable and lasting, and so
we need a way to record their details in some kind of permanent record so they
can be resurrected later.

Examples abound. Consider a social media app, where new users can register,
login, post messages, friend each other, \textit{etc.} We'd likely store
information about all this in various \texttt{User}, \texttt{Post},
\texttt{Profile}, and \texttt{FriendRequest} objects. But it sure would be a
bummer if all that data disappeared any time the server needed to be
restarted!

Or consider a drawing application, which we can use to create figures like
lines, rectangles, circles, and so forth, to create the kinds of diagrams
contained in this book. Our \texttt{Drawing}, \texttt{Line}, and
\texttt{Rectangle} objects, whose instances held information about position,
size, and color, would be next to useless if they weren't able to store
themselves permanently somehow, to be reloaded in a later execution of the
app. It would be like a word processor without a ``save'' function.

What we want is a way to \textbf{persist} (or save) an object, and
\textbf{hydrate} (or restore) it later. Persistence is taking an ephemeral,
in-memory object and writing it out to some form of permanent storage -- a
file in the filesystem, a database, a network drive, or anything else that
will stay put even when the program ends. Hydration is the reverse process:
resurrecting that stored version of the entity's state into a living,
breathing object once again.

\section{Java object serialization}

There are several ways to implement such operations. One, which I don't
recommend, is built into Java and is called ``serialization.'' The
\texttt{java.io} package has classes \texttt{ObjectOutputStream} and
\texttt{ObjectInputStream} for this purpose. The idea is that just as you can
write primitive types like integers and strings to a stream with familiar file
I/O operations, you can also read and write bona fide objects
themselves.\footnote{All objects that are saved and restored in this way must
be from classes that are declared to ``implement'' the
\texttt{java.io.Serializable} ``interface.'' We'll talk about implementing
interfaces in Section~\ref{sec:interfaces}.} Just open an
\texttt{ObjectOutputStream} to a file and call \texttt{.writeObject()}, and
the object you pass, in its entirety, will be saved in the file.
\texttt{ObjectInputStream.readObject()} performs the reverse process.

This all sounds like a great idea, but there's some serious practical
difficulties to consider. For one, \texttt{ObjectOutputStream} stores objects
in a \textbf{binary format} rather than a \textbf{text format}. This means
that the files it produces can't be easily analyzed -- they're compact, but
opaque, sequences of 1's and 0's. If you try to open such a file in a text
editor like \texttt{vim} to inspect what objects and values it contains,
you'll get gobbledy-gook. The upshot is that it's nearly impossible to debug
your program or even get visibility into what was saved.

Just as problematic is the fact that serialization is not
\textbf{forwards-compatible}. When you write an object to a stream in this
way, the data that is persisted is hard-wired to the particular version of the
class the object is a member of. This isn't a problem if your code is
completely, permanently stable. But during a development cycle, you're
constantly changing the nature of many classes, including their instance
variables, names, and types. As soon as you change a class in any significant
way, \textit{bam!!} all previously stored copies of objects are now
unreadable. When you try to hydrate them with \texttt{ObjectInputStream},
it'll break saying, ``whoa, the class of this stored object was an old version
of the class -- it doesn't match the current \texttt{.java} file.'' And that
data is, for all intents and purposes, lost.

I've learned the hard way that when you persist objects, you want to store
them in a format which is transparent and forwards-compatible. You need to be
able to inspect exactly what was stored, and have that data be readable even
if you change the class's definition later on.

\section{Using a relational (or non-relational) database}

An excellent solution for persistence and hydration is to use a Database
Management System (DBMS) like MySQL, PostgreSQL, or MongoDB. These products
specialize in this very task. Some, like those with ``SQL'' in the name,
create \textbf{relational databases} which hold rectangular tables of data
records, sort of like gigantic spreadsheets. Others are called
\textbf{non-relational} or \textbf{NoSQL databases} and can store data in a
more flexible, non-rectangular format. Programs in Java (or any other
language) can create \textbf{connections} to either kind of database, and
read/write to them with standard commands.

The reason I won't go further into this option is because it's really a
separate topic from OOA\&D, and merits a whole course in its own right (at
UMW, it's CPSC 350). Once you learn that material, it's really the best way to
do the whole persistence/hydration thing.

\section{Using plain text files}

In this book, I'll explain how to do persistence and hydration caveman-style:
using plain text files.

