\chapter{Exceptions}

\section{Recognizing error conditions}

The shrewd reader (and driver!) will realize that our \texttt{.drive()} method
from last chapter was a bit optimistic: when told to drive \texttt{numMiles},
it blindly did so, even if there was not enough gas to get that far. We ought
to guard against this kind of wishful thinking by \textit{not permitting} a
drive that's outside our range. If told to drive 1000 miles when we only have
enough gas to go 200, we'll just say no. That's way better than ending up with
a negative gas tank and wreaking unknown havoc later in the program!

The first step in implementing this kind of defensive coding is to figure out
\textit{when} to refuse to carry out orders. That's not too hard in this case:
our local \texttt{galsBurned} variable is exactly what we need: if it turns
out to be higher than the gas remaining in the tank, we are officially in
Nonsense Land. A simple \texttt{if} statement can take care of that.

\section{Suboptimal ways of handling error conditions}

The second step is figuring out \textit{what to do} when this occurs. Most
people's first thought is to blare out a siren:

\begin{Verbatim}[samepage=true,fontsize=\scriptsize,frame=single]
// Inadequate approach #1
class Car {
    ...
    void drive(int numMiles) {
        double galsBurned = numMiles / this.gasMileage;
        if (galsBurned > this.galsRemaining) {
            System.out.println("Not enough gas!!");
        }
        this.galsRemaining = this.galsRemaining - galsBurned;
        this.odo += numMiles;
    }
}
\end{Verbatim}

This is done in the hopes that someone will hear us and be alarmed. The
problem is, this error will go to the console, where it may or may not ever be
seen; and even if someone notices it, we've still already done the dirty deed.
We have a \texttt{Car} object with an \textbf{illegal state}: a negative gas
level.

Slightly better, but still not good enough, is to print the error and also
refuse to carry out orders:

\begin{Verbatim}[samepage=true,fontsize=\scriptsize,frame=single]
// Inadequate approach #2
class Car {
    ...
    void drive(int numMiles) {
        double galsBurned = numMiles / this.gasMileage;
        if (galsBurned > this.galsRemaining) {
            System.out.println("Not enough gas!!");
            return;                                     //  <---  NOTICE THIS!
        }
        this.galsRemaining = this.galsRemaining - galsBurned;
        this.odo += numMiles;
    }
}
\end{Verbatim}

Now, in addition to printing the error, we also \texttt{return} from the
method prematurely, before carrying out the nonsensical operation.

The problem with this approach is that \textit{the client code is not alerted
that anything went wrong.} We'll see some actual client code in action in the
next section, but for now just realize that whoever called
\texttt{.drive(1000)} is none the wiser. It merrily chugs along thinking that
the thousand-mile drive was plain sailing, oblivious to the fact that no such
drive actually occurred.

\section{The right way: \texttt{throw}ing and \texttt{catch}ing}

The right way to handle this is with Java's \texttt{Exception} mechanism. We
don't return prematurely, as above; instead, \textit{we don't return at all.}
Exceptions are Java's way of allowing a method \textit{not} to return, but
rather to raise a big red flag to indicate that carrying out the method just
flat didn't work. It's the only responsible thing to do.

Our first step is to ``throw an exception'' instead of returning. Here's the
code to do so:

\begin{Verbatim}[samepage=true,fontsize=\scriptsize,frame=single]
// Correct approach (not finished yet)
class Car {
    ...
    void drive(int numMiles) {
        double galsBurned = numMiles / this.gasMileage;
        if (galsBurned > this.galsRemaining) {
            throw new Exception("Not enough gas!!");    //  <---  NOTICE THIS!
        }
        this.galsRemaining = this.galsRemaining - galsBurned;
        this.odo += numMiles;
    }
}
\end{Verbatim}

Operationally, throwing an exception has the same immediate effect as
returning: the method instantly terminates and returns control back to the
client code. The difference, as we'll see in the next section, is that the
client code is aware that something unusual happened, does not get a return
value, and can take evasive action.

If you try to compile the above code, though, you get an error, which says:

\begin{Verbatim}[samepage=true,fontsize=\footnotesize]
  error: unreported exception Exception; must be caught or declared to be thrown
\end{Verbatim}

It's actually nice of Java to give this error, and here's why. Inside our
method, we've created a possibility that we won't return at \textit{all}, and
will instead barf with this error. Java requires that if we do that, we 'fess
up and declare that this is a possibility. That prevents unwitting programmers
from blithely calling our method and not accounting for the fact that it might
not even work.

Fixing it is simple; we just change the first line of the function to:

\begin{Verbatim}[samepage=true,fontsize=\scriptsize,frame=single]
    void drive(int numMiles) throws Exception {
        ...
\end{Verbatim}

That first line now says: ``you can call me on a \texttt{Car}, pass me an
integer argument, and get no return value. \textit{But} there's a possibility
that it won't work, and you should be aware of that.'' It's only honest, and
as we'll see, it allows the code that uses \texttt{Car}s to properly deal with
the problem.


\section{Calling a method that \texttt{throws} an \texttt{Exception}}


The only remaining fly in our ointment (don't worry; he's easily swatted) is
that when client code \textit{calls} \texttt{.drive()}, it might not
necessarily run to completion. In fact, if we compile the above main program,
we'll get the same kind of compile error that we did when we were midway
through implementing the Exception-throwing stuff. It'll say we're being
unconscionably remiss by refusing to deal with the error that might occur
every time we tell one of our cars to drive.

The Java way to handle this is with a \textbf{try/catch block}. Essentially,
this just builds a little scaffolding around our call to suspicious methods
like \texttt{.drive()} so that if and exception is indeed ``thrown'' when we
call it, we can ``catch'' is and do something sensible. Here's how:

\begin{Verbatim}[samepage=true,fontsize=\scriptsize,frame=single]
    public static void main(String args[]) {
        ...
        // Caravan to Disneyworld -- whoo-hoo!  (Grammy's meeting us there.)
        minivan.fillUp();
        try {
            minivan.drive(500);
        } catch (Exception e) {
            System.out.println(e);
            System.exit(1);
        }
        ...
    }
\end{Verbatim}

Instead of simply calling \texttt{minivan.drive()} and throwing caution to the
wind, we put that code in a \textbf{try block}. The code in a \texttt{try}
block is executed normally, step-by-step, just like anywhere else in a Java
program. But the \texttt{try} block has one or more (here just one)
contingency plans connected to the bottom of it, which can handle any special
(or ``exceptional'') conditions. In this case, that call to drive the minivan
500 miles through traffic will either work in its entirety and have the
desired effect, or it will abort in the middle of it and control will be
immediately transferred to the relevant catch block below it. The flow
continues in that \textbf{catch block}, in this case printing the message of
the \texttt{Exception} and then terminating the program.

What to do in each exceptional situation depends on the situation itself.
Sometimes, there are meaningful things one can do in response to an error:
like if a network connection fails, the code can retry connecting; or if a
checking account withdrawal fails, the system can cut over to the savings
account and cover the amount from those funds instead. In our case, if our
program tracked things like routes and desired destinations, a caught
exception would indicate that we need to find a new, temporary destination
other than the one we're currently seeking: one that has gas so we can fill
up.

Once all the method calls that are defined as ``\texttt{throws Exception}''
have been enclosed in try/catch blocks that at least nominally handle the
errors, the code compiles again and it can hopefully run error-free.

% TODO: Put this chapter after "Memory matters" and include "how to interpret
% a stack trace".

\begin{figure}
\begin{Verbatim}[fontsize=\scriptsize,frame=single]
class Car {
    String make, model;
    int yearsOld, odo;
    double galsRemaining, sizeOfTank, gasMileage;

    Car(String make, String model) {
        this.make = make;
        this.model = model;
        yearsOld = 0;
        odo = 0;
        galsRemaining = 0;
        if (make.equals("Chevy") || make.equals("GM")) {
            sizeOfTank = 21;
        } else {
            sizeOfTank = 13;
        }
        if (make.equals("Chevy") && model.equals("Malibu")) {
            gasMileage = 3;
        } else {
            gasMileage = 24;
        }
    }

    public String toString() {
        return "a " + yearsOld + "-year-old " + make + " " + model;
    }

    String getMake() { return make; }

    String getModel() { return model; }

    int getYearsOld() { return yearsOld; }

    void setYearsOld(int x) { yearsOld = x; }

    void fillUp() {
        this.galsRemaining = this.sizeOfTank;
    }

    double getTankPerc() {
        double perc = galsRemaining / sizeOfTank * 100;
        return perc;
    }

    void drive(int numMiles) throws Exception {
        double galsBurned = numMiles / gasMileage;
        if (galsBurned > galsRemaining) {
            throw new Exception("Not enough gas!");
        }
        galsRemaining = galsRemaining - galsBurned;
        odo += numMiles;
    }
}
\end{Verbatim}
\caption{A complete Java implementation of the Car class, with exception
handling added.}
\label{fig:carClassCode}
\end{figure}

