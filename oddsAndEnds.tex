
\chapter{Java Odds 'n' Ends}

Before we continue our study of OOA\&D proper, let's look at a few
Java-specific idiosyncrasies which will be all up in our business soon enough.

\section{Garbage collection}

No, I didn't make that term up just to be funny. \textbf{Garbage collection}
is actually the official name for a Java feature which was super innovative at
the time, but which we now often take for granted.

Consider the code for a \texttt{Ball} class given in
Figure~\ref{fig:ballCode}. When we run it, Java calls \texttt{main()}, which
calls \texttt{play()}. At the end of \texttt{play()}, right before it returns,
a memory diagram would look like Figure~\ref{fig:ballMemory}. Take a moment to
see if you agree with all the details.

\begin{figure}[ht]
\begin{Verbatim}[fontsize=\scriptsize,samepage=true,frame=single]
class Ball {
    private String color;
    private int airPressure;

    Ball(String color) {
        this.color = color;
        airPressure = 0;
    }

    void bounce() {
        System.out.println("Boing!!");
    }

    static Ball play(int numBalls) {
        ArrayList equipment = new ArrayList();
        Ball b;
        int i;
        for (i=0; i<numBalls; i++) {
            b = new Ball("red");
            b.bounce();
            equipment.add(b);
        }
        Ball basket = new Ball("orange");
        return basket;
    }

    public static void main(String args[]) {
        int x = 3;
        Ball myBall = play(2);
        System.out.println("My ball is " + myBall.color + ".");
    }
}
\end{Verbatim}
\caption{A class to illustrate the utility of garbage collection.}
\label{fig:ballCode}
\end{figure}

At this moment, we have an active stack frame for the \texttt{play()} function
which contains five variables of various types. And we're getting ready to
return the reference variable \texttt{basket} back to \texttt{main()}, which
means that about a nanosecond from now \texttt{main()} will be assigning its
new \texttt{myBall} variable to point to that orange ball.

\begin{figure}[ht]
\centering
\includegraphics[width=1\textwidth]{ballMemory.pdf}    % 750x400
\caption{A snapshot of memory, taken just before the \texttt{play()} function
returns back to \texttt{main()}.}
\label{fig:ballMemory}
\end{figure}

Okay, now let's do it. We return to \texttt{main()}. As soon as we do, the
memory diagram looks like Figure~\ref{fig:ballMemory2}. Take a close look.
That second diagram is all correct, but something about it may strike you as a
bit weird; namely, there are three objects on the heap \textit{with nothing on
the stack referencing them}. The \texttt{myBall} variable dutifully points to
the orange ball that was returned, but the two red balls, and the
\texttt{ArrayList} that contained them, are now disembodied from everything
else. And in fact, they are effectively \textit{lost} to the program. There's
simply no way to reference them. 

\begin{figure}[ht]
\centering
\includegraphics[width=1\textwidth]{ballMemory2.pdf}    % 750x400
\caption{The state of memory right \textit{after} the return to
\texttt{main()}. Notice there are now three unreachable objects on the heap.}
\label{fig:ballMemory2}
\end{figure}

If you're unsure that there's truly no way, ask yourself this question: ``what
line of code could we write to (say) change one of the \texttt{Ball} object's
color from red to blue?" The answer is: there is no possible line of code we
could write to do that. To even get off the ground we'd have to start with a
\textit{name}, and there is no name we could possibly use to get at either of
those red \texttt{Ball} objects.

Now with C++, a language that preceded Java by decades, this would be a bad
situation called (I kid you not) a \textbf{memory leak}. The memory the
program used to store those now-unreachable \texttt{Ball}s is now inaccessible
to the program, and what's worse, \textit{C++ doesn't realize that's the
case.} So those old objects just sit there, growing stale, occupying system
memory that could be used to store other things instead. The program never
realizes this, and so never reclaims that space. So the actual amount of
memory the program has available to it has effectively shrunk.

In C++, the only remedy for this situation is for the programmer herself to
keep track of which objects no longer have any stack references to them, and
to explicitly \texttt{delete} those objects. This is a delicate task: fail to
\texttt{delete} what is in fact delete-able and you'll have a memory leak;
eagerly \texttt{delete} what actually does have other references to it and
your program will crash when that memory is reused by something else writing
over the top of it. The whole situation is fraught with peril.

Enter Java, in 1995. Java featured \textbf{automatic garbage collection} which
outsourced the whole responsibility for this from the programmer to a special
Java background task called the \textbf{garbage collector}. Whenever the
garbage collector runs, it intelligently sifts through the contents of memory,
looking for junk that can't be legally accessed anymore anyway. Whenever it
finds such junk, it \textit{automatically} tells the memory manager that that
memory is no longer in use, and can therefore be repurposed the next time the
program requests some memory with \texttt{new}.

Automatic garbage collection is lit. It means that pictures like
Figure~\ref{fig:ballMemory2} aren't scary at all. Sure, we have three objects
in the heap that can't ever be reached, but the garbage collector will soon
run, figure that out, and reacquire that memory so it can be used again. All
this without the programmer having to think a thing about it. Memory leaks are
in principle a thing of the past.

As with most good things, there are downsides too. One downside to Java's
approach is that the garbage collector thread decides to run ``any time it
damn well pleases." As developers, we don't ever tell Java to get off its butt
and take out the trash (although there is a way to suggest this); rather, we
just wait for it to run when it periodically thinks it needs to. This isn't
normally a problem, but it can be in mission-critial real-time applications
with absolute performance deadlines. Think of the software running on a
pacemaker, which is embedded in a human heart. The code \textit{must} respond
in a certain amount of time in order to trigger the heart to take its next
beat! Now if, during our pacemaker program's operation, the garbage collector
suddenly decided to run in order to clean up lost memory, it might be a
time-consuming operation in its own right. And our program could literally
skip a beat (or beat later than it should) while waiting for it to finish. In
situations like these, C++'s style of manual control does give us more
fine-tuned flexibility over when exactly the reclamation of lost memory
occurs. For most applications, though, we don't need that flexibility and
Java's way of handling it is much appreciated.

\section{The \texttt{import} statement}

You've probably used \texttt{import} in almost every Java program you've ever
written. Yet I've found most developers don't really understand what it does.
Java itself is partially to blame here: the word ``import" is a poor choice of
words, since nothing is ``imported" at all.\footnote{Also, the fact that it
begins with a lower-case \texttt{i} just like C++'s ``\texttt{\#include}"
statement does reinforces this misconception -- C++'s \texttt{\#include}
actually \textit{does} include/import content.}

Here's a statement that surprises a lot of Java programmers: you can write
\textit{any} Java program -- even one that uses stuff in the Java API --
\textit{without} an \texttt{import} statement. The left side of
Figure~\ref{fig:dontNeedImport} is such a program.

\begin{figure}[ht]
\begin{Verbatim}[fontsize=\scriptsize,samepage=true,frame=single]
                                                  import java.util.ArrayList;

class YouDontNeedImport {                         class YouDontNeedImport {                                                 
  public static void main(String args[]) {          public static void main(String args[]) {
    java.util.ArrayList celebs =                      ArrayList celebs =
      new java.util.ArrayList();                        new ArrayList();
    celebs.add("Kim Kardashian");                     celebs.add("Kim Kardashian");
    celebs.add("Justin Bieber");                      celebs.add("Justin Bieber");
    celebs.add("Taylor Swift");                       celebs.add("Taylor Swift");
    printEm(celebs);                                  printEm(celebs);
  }                                                 }
                                                
  static void printEm(java.util.ArrayList l) {      static void printEm(ArrayList l) {
    for (int i=0; i<l.size(); i++) {                  for (int i=0; i<l.size(); i++) {
      System.out.println("People love "                 System.out.println("People love "
        + l.get(i));                                      + l.get(i));
    }                                                 }
  }                                                 }
}                                                 }
\end{Verbatim}
\caption{The same Java program, using explicit inline package names (left),
and the \texttt{import} statement (right).}
\label{fig:dontNeedImport}
\end{figure}

No \texttt{import} required. Instead, every time we wanted to refer to the
\texttt{ArrayList} class from the Java API, we specified it as
\texttt{java.util.ArrayList} That did require us to type out the full name
three times, but it turns out to be all Java needs to understand perfectly
which \texttt{ArrayList} class we want to use.

Now since programmers (myself included) are lazy, and want to avoid typing
when possible, the Java gods invented the \texttt{import} statement. The
\textit{only} thing it does is tell Java ``I don't really feel like typing out
\texttt{java.util.ArrayList} every time. That's a pain. So Java, please know
that when I type \texttt{ArrayList} in this file, I really mean
\texttt{java.util.ArrayList}."

The right-side of Figure~\ref{fig:dontNeedImport} is exactly the same program,
but now using the \texttt{import} statement to avoid a little typing.

Whether the savings are worth it in any particular case is up to you. My point
here is just to demonstrate that the statement ``\texttt{import
java.util.ArrayList}" does \textit{not} do anything remotely like ``go and
find the \texttt{ArrayList} code in the \texttt{java.util} package, and bring
it in here so I can use it." Nor is it true that ``unless you import that
class, you can't use it." Both are common Java myths.

By the way, you may have seen the use of \texttt{*} syntax, like this:

\begin{Verbatim}[fontsize=\small,samepage=true]
import java.util.*;
import com.google.search.engines.*;
import edu.umw.stephen.coolclasses.*;
\end{Verbatim}

This isn't a good idea. The reason is that it's ambiguous. Suppose in my code
I refer to a class called \texttt{Scanner}. Which \texttt{Scanner} do I mean?
Should Java assume I wanted to avoid typing ``\texttt{java.util.Scanner}" or
``\texttt{com.google.search.engines.Scanner}" or
``\texttt{edu.umw.stephen.coolclasses.Scanner}"? It has no way of knowing. So
if more than one of those packages defines a class with the same name (which
is very possible), Java will at best not compile, and at worst give unexpected
runtime behavior.

For this reason, using ``\texttt{*}" in \texttt{import} statements is only
acceptable when writing quick and dirty one-off code, not for anything that
will stick around longer than your current coding session.

\section{``Wrapper" classes}


\section{Java ``generics"}

\section{The \texttt{Hashtable} data structure}

\section{Sameness vs.~identicality}
