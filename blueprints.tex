
\chapter{Blueprints: UML class diagrams}

We spent last chapter discussing the \textbf{dynamic} view of a program: what
happens to memory, step by step, as it unfolds. In this chapter, we'll switch
to a \textbf{static} view: long-term, what are the program's classes, methods,
and relationships between them?\footnote{The words ``dynamic" and ``static"
are ubiquitous in computer science, and mean a zillion different unrelated
things. For example, we've already seen the Java ``\texttt{static}" keyword,
and how it indicates class-level rather than an object-level ownership. We've
also hinted at the stack having ``statically-allocated memory" and the heap
being ``dynamically-allocated." These terms are \textbf{\textit{unrelated}} to
our use of the words in this chapter. At present, by ``dynamic" we mean ``the
contents of memory changing as the program runs"; and by ``static" we mean
``the consistent, permanent characteristics of a program, quite apart from how
it might be behaving at any moment, which include its classes, methods, and
associations."}

If there's a type of UML diagram that deserves the name ``blueprint," it's the
\textbf{class diagram}. Class diagrams depict a high-level, stable perspective
of a software system. When you want to figure out how a large OO program
works, or when you're tasked with implementing a system that someone else has
designed, the first thing you look at are its class diagram(s).

UML class diagrams contain a number of elements, each of which has a very
specific meaning. We'll cover each in turn.

\section{Classes}

Unlike memory diagrams, which depict objects, class diagrams contain classes
(duh). We've already seen what a single class looks like in
section~\ref{sec:UMLclasses} (\textit{e.g.}, the left side of
Figure~\ref{fig:classObject}.) Most class diagrams contain many such classes.
Recall that each class has three compartments, containing the class's name,
its inst vars, and its methods, in that order.

By the way, one flexible (yet slightly annoying IMO) aspect of UML is that it
allows \textbf{varying levels of detail}. In other words, on a particular
diagram, you may or may not want to show all the instance variables and
methods, because it may or may not be relevant to the purpose of that
particular diagram. Similarly, you may or may not want to show all the aspects
of each inst var or method; perhaps it's too early in the design process to
completely specify all the parameters and return types, for example. To
illustrate, all three pictures in Figure~\ref{fig:graceful} are legit
ways of representing the \texttt{Car} class. We can include as much or as
little detail as we please.

\begin{figure}[ht]
\centering
\includegraphics[width=1\textwidth]{graceful.pdf}   % 670x320
\caption{Three equally valid ways to draw the \texttt{Car} class on a class
diagram, depending on how much detail it makes sense to include.}
\label{fig:graceful}
\end{figure}

The reason I find this annoying, by the way, is that it's ambiguous. If you
see no inst vars in the second box, does that mean (a) that class \textit{has}
no inst vars, or (b) the designer didn't think it was relevant to include them
on this particular diagram? No way to really know.

\section{Associations}

Perhaps the most important bits of information on a class diagram are the
\textbf{association}s between classes. An association means that two classes
collaborate together in some way to achieve some larger purpose. It is
indicated on a class diagram by a line connecting the two classes. Different
types of lines represent different kinds of relationships between the classes.
It's important not to mix them up, because if you do, you're dictating
something incorrect to the programming team about how the classes are intended
to work.

\begin{figure}[ht]
\centering
\includegraphics[width=0.7\textwidth]{assocArrows.pdf}   % 350x200
\vspace{.2in}
\caption{Diagrammatic elements for different association types.}
\label{fig:assocArrows}
\end{figure}

\subsection{Dependency associations}

Figure~\ref{fig:assocArrows} shows some of the UML association arrows and
their meaning. (There are others we'll get to in future chapters.) The dashed
line with a crow's foot arrowhead is called a \textbf{dependency}, and is the
``weakest" of the association types. When I say weak, I mean that the
relationship between the two classes isn't as important, nor as permanent, as
with the other association types we'll discuss later.

A dependency between classes \texttt{A} and \texttt{B} can be thought of in a
couple of ways:

\begin{compactitem}
\item One or more methods of the \texttt{A} class will \textit{call methods
on} a \texttt{B} object.
\item The \texttt{A} class \textit{is dependent on the interface of} the
\texttt{B} class.
\end{compactitem}

The word \textbf{interface} -- like stack, heap, dynamic, static, and many
other computer science words -- has multiple meanings. We'll talk about the
Java \texttt{interface} keyword later in the book. For now, when I say
interface I mean \textit{those aspects of a class that a user of that type of
object can see.} This boils down to: the methods you can call on it, together
with their argument lists and return types. Specifically, the interface does
\textit{not} include the method implementations (the bodies of the methods),
nor the instance variables.

If you think about it, you'll realize why the above two bullet points are
actually equivalent. Suppose some class \texttt{A} method has this line of
code in it: ``\texttt{String s = B.scissorKick(15)}". Then clearly the code in
the \texttt{A.java} file is \textit{dependent} on the fact that class
\texttt{B} has a \texttt{.scissorKick()} method, and that it takes an integer,
and returns a \texttt{String}. If any of that ever changed in the
\texttt{B.java} file, then class \texttt{A} would be impacted.

\begin{figure}[ht]
\centering
\includegraphics[width=0.8\textwidth]{dependencyAssoc.pdf}   % 
\caption{Examples of dependency associations.}
\label{fig:dependencyExamples}
\end{figure}

The strange-looking words adjacent to the dependency arrows in
Figure~\ref{fig:assocArrows} go by the even stranger-sounding term
\textbf{stereotypes}. A stereotype in UML is an extra bit of information that
enhances part of a diagram (an association arrow, as here, or sometimes a
class, method, or other element) by making its meaning more precise.
Stereotypes are usually displayed enclosed by double-wakkas
(``$\ll$...$\gg$").

In the case of dependency associations, the stereotype ``$\ll$uses$\gg$" means
pretty much what a dependency always means: that the designer intends class
\texttt{A} to ``use" (\textit{i.e.}, get its hands on, and call method(s) on)
object(s) of class \texttt{B}. The ``$\ll$instantiates$\gg$" stereotype goes a
bit further, and implies that some method of \texttt{A} will
\textit{instantiate} \texttt{B} objects in addition to merely calling methods
on them.

The examples in Figure~\ref{fig:dependencyExamples} are from a Dungeons \&
Dragons type combat simulator. A \texttt{Battle} object represents a fight
between adventurers and monsters. While simulating this fight, a
\texttt{Battle} will make use of one or more \texttt{Die} (singular of ``dice")
objects to roll random numbers that determine the outcome. This is a
``$\ll$uses$\gg$" association, since \texttt{Battle}'s code now depends on
\texttt{Die}'s interface not changing.

Elsewhere in the program, wizards sometimes cast ranged spells, like fireballs
or lightning bolts, to damage distant enemies. In the simulator, a
\texttt{Wizard} object might therefore instantiate a \texttt{RangedSpell}
object to carry out this attack. Since somewhere in the \texttt{Wizard} class's
code there will be a ``\texttt{new RangedSpell()}" line, we say that
\texttt{Wizard} $\ll$instantiates$\gg$ \texttt{RangedSpell}.


\subsection{``Has-a" associations}

The next strongest type of association has a bizarre name: it's called
``\textbf{has-a}." We denote it with a solid arrow between classes, with a
crow's foot on one side or both.

When class \texttt{A} has-a class \texttt{B}, that is nearly always a signal
to the programmer that \texttt{A} should have an \textit{instance variable} of
type \texttt{B}.\footnote{Or perhaps a \textbf{collection} of \texttt{B}
objects rather than a single \texttt{B} object, as we'll see later in the
chapter.} In other words, not only does an \texttt{A} object call
methods on a \texttt{B} (as in the dependency association), but an \texttt{A}
object actually holds on to one (or more) \texttt{B} objects for the
long-haul.

Now in some cases, the ``has-a" verbiage makes perfect sense. If our Domino's
Pizza delivery manager application had a \texttt{Pizza} class and a
\texttt{Topping} class, it would be no-brainer to say that every
\texttt{Pizza} has-a \texttt{Topping}. It conjures up in our minds a picture
of containment, or ownership. Perfect. However, we also use this type of
association in cases where containment doesn't make sense at all.

For example, in the same application it would be quite sensible to say that
``every \texttt{Pizza} has-a \texttt{DeliveryCar}." But obviously the delivery
car isn't ``inside" the pizza in the same physical way that the toppings are
inside it. So what does it mean then?

\begin{figure}[ht]
\centering
\includegraphics[width=1\textwidth]{wrongRightHasA.pdf}   % 
\caption{The \textbf{wrong}, and \textbf{right}, way of visualizing a ``has-a"
association in Java.}
\label{fig:wrongRightHasA}
\end{figure}

The key is making sure you have the right mental model.
Figure~\ref{fig:wrongRightHasA} shows both the wrong, and the right, way to
envision a has-a relationship (at least, in Java). In memory, there is
\textit{no} ``containment" as in the left-hand (wrong) image. The
\texttt{Topping} object isn't enclosed inside the \texttt{Pizza}, or even
exclusively owned by it. It's simply pointed to by one of the \texttt{Pizza}
object's inst vars. The right-hand side of the figure is the correct one --
and I daresay it's not problematic at all to think of a \texttt{Pizza}
``having" a \texttt{DeliveryCar} in this way. All it really means is that a
\texttt{Pizza} object ``knows about" a \texttt{DeliveryCar}, which is the
particular car that's delivering it.

Another reason that the correct mental model of ``has-a" is important is that
it is possible, and even common, for the association to go \textit{both ways}.
We use the term \textbf{navigability} for the question ``which direction does
the arrow go -- from \texttt{A} to \texttt{B}, from \texttt{B} to \texttt{A},
or both?" When it goes both ways, we call it a \textbf{bidirectional}
association.

\begin{figure}[ht]
\centering
\includegraphics[width=1\textwidth]{bidirectional.pdf}   % 750x235
\caption{A bidirectional ``has-a," depicted on a class diagram (left) and a
memory diagram.}
\label{fig:bidirectional}
\end{figure}

An example is the left-hand side of Figure~\ref{fig:bidirectional}. Here, our
\texttt{Driver} class and our \texttt{DeliveryCar} class each know about the
other, and in fact each hold on to an instance variable of the other type. If
we viewed this \texttt{A}-having-an-instance-variable-of-type-\texttt{B} thing
as the \texttt{A} object \textit{enclosing} the \texttt{B}, we'd blow a fuse.
\texttt{A} would contain \texttt{B}, which would contain \texttt{A}, which
would contain \texttt{B}, which... That way madness lies. But notice that
nothing paradoxical happens at all in the corresponding memory diagram on the
right-hand side of the figure. Each object points to the other, so that a
\texttt{Driver} object knows which \texttt{DeliveryCar} he/she is driving, and
a \texttt{DeliveryCar} also knows which \texttt{Driver} is driving it. No
biggie.

Note, by the way, that the has-a arrow implies the existence of the inst var
\textit{all by itself}. The class diagram should \textit{not} contain a
duplicate copy of the inst var in its second compartment. That would be
redundant, and is considered an error (see Figure~\ref{fig:wrongHasA}).

\begin{figure}[ht]
\centering
\includegraphics[width=1\textwidth]{wrongHasA.pdf}   % 640x180
\caption{One wrong way to model an instance variable. The ``has-a" arrow
already indicates that every \texttt{Pizza} has-a \texttt{Topping}: the
extraneous \texttt{topping} entry in the \texttt{Pizza} class's second box is
redundant and incorrect.}
\label{fig:wrongHasA}
\end{figure}
