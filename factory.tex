
\chapter{The Factory Pattern}

Now that we've covered inheritance, we're in a position to understand the next
simple-yet-ubiquitous design pattern, called \textbf{Factory}. It's a pretty
easy one to grasp. Simply put, a factory is \textit{a class whose purpose is
to instantiate objects.}

Up to now, we've used the \texttt{new} operator directly in order to
instantiate. If we want a new \texttt{Ballplayer} object, we \texttt{new} one
up:

\begin{Verbatim}[fontsize=\small,samepage=true,frame=none]
  Ballplayer joe = new Ballplayer("Dimaggio","OF");
\end{Verbatim}

We're now going to outsource this instantiation process to a special class
called a \texttt{BallplayerFactory}. It'll seem like unnecessary wiring at
first, but there are advantages that come to light when the instantiation
process is more complicated. Our new line of code will be this:

\begin{Verbatim}[fontsize=\small,samepage=true,frame=none]
  Ballplayer joe = BallplayerFactory.instance().create("Dimaggio","OF");
\end{Verbatim}

The sharp-eyed reader will see the \texttt{.instance()} and wonder if this is
using the \textbf{Singleton} pattern. The answer is yes! \textit{Factories are
most often singletons.} This is simply because although our baseball simulator
will create lots of \texttt{Ballplayer}s, it will only need one
\texttt{BallplayerFactory}.\footnote{As with all design patterns, you should
use standard nomenclature. If the word ``factory" seems clunky or contrived to
you, I get it, but accept it as part of the OOP culture and use it. Don't try
to come up with your own synonym of ``factory" (like ``creator" or
``instantiator") and use that instead, since you'll just confuse your fellow
programmers. The word ``factory," like the word ``singleton," is baked into
the software development community's consciousness, and thus serves as an
excellent terse-yet-precise communication of purpose.}

Patterns interleave and play off one another all the time. Here, we have a
factory class that uses the classic Singleton pattern:

\begin{Verbatim}[fontsize=\small,samepage=true,frame=single]
class BallplayerFactory {

    private static BallplayerFactory theInstance;
    
    public static synchronized BallplayerFactory instance() {
        if (theInstance == null) {
            theInstance = new BallplayerFactory();
        }
        return theInstance;
    }

    private BallplayerFactory() {
    }

    public Ballplayer create(String name, String position) {
        return new Ballplayer(name, position);
    }
}
\end{Verbatim}

Badda-boom.

Now I know what you're thinking. You're thinking ``wow, that's a whole lot of
code to do nothing but ``\texttt{new} up a \texttt{Ballplayer}," which we were
doing before with one line of code and a lot less work. What's the point of
all this?"

The answer is that oftentimes \textit{the code which needs to instantiate an
object isn't aware of what specific subtype of object it needs.} Read that
sentence again and let it sink in. This type of decoupling -- one part of the
code that directs objects to do things without being aware of subclasses, and
a different part of the code that implements subtype-specific behavior -- is
one of the important benefits a properly-designed object-oriented program can
bring.

\section{Top-down inheritance and the Factory pattern}

This is in fact the launch point of top-down inheritance. We've seen that with
top-down inheritance, the client code can treat all specific subtypes (like
\texttt{Cow}, \texttt{Duck}, \textit{etc.}) in exactly the same way -- in
fact, it doesn't even have to be aware that there \textit{are} any subtypes.
To the client code, all that exists are \texttt{Animal}s, and those objects
can be directed to move, make noise, or anything else.

That was all great, but one thing we didn't address was ``how do those objects
come into existence in the first place?" Somewhere the specific subtype
\textit{has} to be mentioned in the code, or else there's no way to
``\texttt{new}" it. How can we be blissfully ignorant of the subtypes if we're
the one who has to instantiate them?

The answer is the Factory pattern. With it, we turn over control of the actual
instantiation to the factory class, rather than burdening the client code with
it.

Suppose that our simulator is more sophisticated than the toy example from
Chapter~\ref{ch:memoryMatters}. Perhaps different positions have different
kinds of stats: pitchers (with ``earned-run average" (ERA) and ``innings
pitched") are completely different than position players (who have batting
averages and runs-batted-in (RBI's) instead). It might make sense to use an
inheritance hierarchy here with subclasses of \texttt{Ballplayer}, as shown in
Figure~\ref{ballplayerInheritance}.

\begin{figure}[ht]
\begin{Verbatim}[fontsize=\small,samepage=true,frame=single]
class Ballplayer {
    protected String name;
    private int uni;
    private int salary;
    private String handedness;  //  "L" or "R"
}

class Pitcher extends Ballplayer {
    private double era;
    private double inningsPitched;

    public Pitcher(String name) { this.name = name; }
}

class PositionPlayer extends Ballplayer {
    private String position;
    private double battingAvg;
    private int rbis;

    public PositionPlayer(String name, String pos) {
        this.name = name;
        this.position = pos;
    }
}
\end{Verbatim}
\caption{A small inheritance hierarchy for baseball players.}
\label{ballplayerInheritance}
\end{figure}

Our factory can now instantiate \textit{the right kind} of \texttt{Ballplayer}
when it is asked to:

\begin{Verbatim}[fontsize=\small,samepage=true,frame=single]
class BallplayerFactory {
    ...

    public Ballplayer create(String name, String position) {
        if (position.equals("P")) {
            return new Pitcher(name);
        } else {
            return new PositionPlayer(name, position);
        }
    }
}
\end{Verbatim}

Depending on the position (``\texttt{P}" for pitcher, ``\texttt{OF}" for
outfielder, \textit{etc.}) the factory instantiates the proper subtype of
\texttt{Ballplayer} and returns it. The client code doesn't need to know
anything about those subtypes. Notice that the return value of
\texttt{.create()} is the general type \texttt{Ballplayer}, not any of the
specific types. If we later add additional subclasses, only the factory class
has to be updated, not the code that (unknowingly) uses them.

